% русский язык и переносы
\usepackage[OT1]{fontenc}
\usepackage{txfonts}
\usepackage{xunicode}
\usepackage{xltxtra}
\usepackage{xecyr}
\usepackage{textcomp}

% Основной шрифт
\setmainfont[Mapping=tex-text]{Times New Roman}
% Моноширинный шрифт
%\setmonofont[Scale=MatchLowercase]{Courier New}
\newcommand{\listingfont}{\fontspec{Consolas}}
% Стандартные сочетания символов ---, --, << >> и т.п.
\defaultfontfeatures{Mapping=tex-text}
% Переносы в русских текстах
\newfontfamily\russianfont{Times New Roman}
% Для переноса составных слов
\XeTeXinterchartokenstate=1
\XeTeXcharclass `\- 24
\XeTeXinterchartoks 24 0 ={\hskip0pt}
\XeTeXinterchartoks 0 24 ={\nobreak}

%\setmainfont[Mapping=tex-text]{GOST type A}

% Устанавливаем шрифты "ГОСТ А" и "ГОСТ Б" в рамках ЕСКД
\renewcommand{\ESKDfontShape}{\fontspec{GOST type A Italic}}
