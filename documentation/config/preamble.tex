\documentclass[a4paper,utf8]{eskdtext}

\newcommand{\No}{\textnumero} % фикс для новый версий какого-то пакета

\usepackage{graphicx} % графика
\usepackage{eskdtotal} % подсчёт всякой всячины
\usepackage{listings} % исходники
\lstdefinestyle{sharpc}{language=[Sharp]C}
\lstset{
	basicstyle=\fontsize{3.5mm}{0.5em}\listingfont,
	breaklines=true,
	tabsize=2,
	keepspaces=true,
	style=sharpc
}

% ещё 2 уровня вложенности!
\usepackage{eskdpara}
\usepackage{tocloft}
\setcounter{secnumdepth}{5}
\setcounter{tocdepth}{5}

% кое-что с содержанием
\renewcommand{\cfttoctitlefont}{\hfil \bfseries\Large\MakeUppercase}
\renewcommand{\cftaftertoctitle}{\thispagestyle{empty}}
\cftsetindents{section}{0.5in}{0.25in}
\cftsetindents{subsection}{1.0in}{0.3in}
\cftsetindents{subsubsection}{1.5in}{0.5in}
\cftsetindents{paragraph}{2.0in}{0.7in}

% таблицы
\usepackage{longtable} 
\usepackage{multirow}
\usepackage{makecell} % для заголовков таблиц
\newcommand{\theadnum}[1]{\makecell[c]{\footnotesize #1}}
\renewcommand\theadfont{\bfseries}

\newenvironment{myTable}
{ \vspace{3mm}\setstretch{1.0}\begin{small} }
{ \end{small} }

\hyphenation{УлГТУ}

% другое для текста
\usepackage{textcase} % для uppercase
\usepackage{amssymb,amsfonts,amsmath,amsthm} % математика
\usepackage{enumerate} % перечислялки
\usepackage{pdfpages} % для подключения pdf

% шрифты
% Основной шрифт
\setmainfont[Mapping=tex-text]{Times New Roman}
% Моноширинный шрифт
%\setmonofont[Scale=MatchLowercase]{Courier New}
\newcommand{\listingfont}{\fontspec{Consolas}}
% Стандартные сочетания символов ---, --, << >> и т.п.
\defaultfontfeatures{Mapping=tex-text}
% Переносы в русских текстах
\newfontfamily\russianfont{Times New Roman}
% Для переноса составных слов
\XeTeXinterchartokenstate=1
\XeTeXcharclass `\- 24
\XeTeXinterchartoks 24 0 ={\hskip0pt}
\XeTeXinterchartoks 0 24 ={\nobreak}

%\setmainfont[Mapping=tex-text]{GOST type A}

% Устанавливаем шрифты "ГОСТ А" и "ГОСТ Б" в рамках ЕСКД
\renewcommand{\ESKDfontShape}{\fontspec{ГОСТ тип А Italic}}
\usepackage[font=small]{caption}

% интервалы и отступы
\ESKDsectSkip{section}{4em}{4em}
\ESKDsectSkip{subsection}{3em}{2em}
\ESKDsectSkip{subsubsection}{2em}{2em}
\setlength\parindent{1.27cm}
\setlength{\ESKDexplanIndent}{-25pt} % фикс для описания формул
\renewcommand{\intextsep}{0pt}

% межстрочные интервалы
\usepackage{setspace}
\setstretch{1.5}
%\renewcommand{\baselinestretch}{1.5}

% заголовки
\ESKDsectStyle{section}{\bfseries\Large\centering\MakeUppercase}

%\usepackage{ulem} % для умных подчёркиваний

% некоторые графы рамки
\ESKDdepartment{МИНИСТЕРСТВО ОБРАЗОВАНИЯ И НАУКИ РОССИЙСКОЙ ФЕДЕРАЦИИ}
\ESKDcompany{федеральное государственное образовательное учреждение высшего образования \\ 
Ульяновский Государственный Технический Университет}
\ESKDdocName{\WorkName} % (графа 1)
\ESKDauthor{\AuthorName}
\author{\AuthorName}
\ESKDchecker{\ManagerName}
\renewcommand{\ESKDcolumnXfIVname}{Реценз.}
\ESKDcolumnXIfIV{\ReviewerName}
\ESKDapprovedBy{\ControllerName}
\ESKDnormContr{\NormControllerName}
\ESKDletter{У}{Р}{} % учебный реальный (графа 4)
\ESKDgroup{\AuthorGroup} % учебная граппа (графа 9)
\ESKDsignature{\WorkNumber~ПЗ} % обозначение документа (по п.6.1.1) 
\ESKDdefaultTitleStyle{empty}
\renewcommand{\ESKDtheTitleFieldX}{Ульяновск, 2017}

% информация о работе
% === Фамилии ===

\newcommand{\AuthorName}{Моисеев В.В.}
\newcommand{\ManagerName}{Воронина В.В.}
\newcommand{\NormControllerName}{Тимина И.А.}
\newcommand{\ReviewerName}{Кандаулов В.М.}
\newcommand{\ControllerName}{Афанасьева Т.В.}

% === Названия, номера ===
\newcommand{\AuthorGroup}{ПИмд-21}
\newcommand{\WorkName}{Организация межмодульного и межсистемного информационного обмена ИС~ЖКХ}
\newcommand{\WorkNumber}{МД-УлГТУ-09.04.04-15/991-2017}

% === Мелочь ===
%\newcommand{\WorkYear}{2015}
%\newcommand{\WorkPlace}{Ульяновск}

% списки
\usepackage[shortlabels]{enumitem}
\setlist[enumerate]{nosep,leftmargin=2cm} % ,label*=\arabic*.
\AddEnumerateCounter{\Asbuk}{\@Asbuk}{\CYRM}
\AddEnumerateCounter{\asbuk}{\@asbuk}{\cyrm}

\newlength{\wideitemsep}
\setlength{\wideitemsep}{1.5\itemsep}
\let\olditem\item
\renewcommand{\item}{\setlength{\parsep}{100pt}\olditem} % ???

% ещё немного списков
\usepackage[ampersand]{easylist}
\ListProperties(Hide=100, Hang=true, Progressive=1.27cm, Style*=-- , Style2*=$\bullet$, Style3*=$\circ$, Style4*=\tiny$\blacksquare$ )

% рисунки
\usepackage{flafter}% помещает флоат ПОСЛЕ первой ссылки на него

\newcommand{\myFigure}[5][h!]{
	\begin{figure}[#1]
		\vspace{2em}
		\begin{minipage}[h]{\linewidth}
			\centering
			\includegraphics[width=#2\linewidth]{images/#3}
			\vspace{1mm}
			\caption{#4}
			\label{img:#3}
		\end{minipage}
	\end{figure}  
}