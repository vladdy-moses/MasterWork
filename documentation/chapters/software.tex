\section{Программное обеспечение модели}

\subsection{Модуль интеграции РИАС ЖКХ с ГИС ЖКХ}

\subsubsection{Общее описание}

Был разработан модуль интеграции <<РИАС ЖКХ. Модуль интеграции с ГИС ЖКХ>>.

Модуль позволяет проводить двусторонний обмен следующими видами информации:
\begin{easylist}
& договоры управления и уставы;
& договоры ресурсоснабжения;
& сведения о домах и помещениях;
& лицевые счета;
& приборы учёта и их показания;
& платёжные документы;
& факты оплат и отзыв платежей;
& перечни работ управляющих организаций;
& проверки ГЖИ и планы проверок.
\end{easylist}

Дополнительно в модуле интеграции с ГИС ЖКХ реализована следующая функциональность:
\begin{easylist}
& получение из ГИС ЖКХ реестра организаций согласно ЕГРЮЛ/ЕГРИП;
& получение из ГИС ЖКХ нормативно-справочной информации (НСИ);
& двусторонний обмен файлами;
& TODO: дополнить.
\end{easylist}

Программное обеспечение использует следующие технологии информационного взаимодействия:
\begin{easylist}
& SOAP по зашифрованному (алгоритм ГОСТ 34.TODO TODO: ссылка) каналу связи с подписью бизнес-данных по xades-bes;
& разбор xml-ответов от веб-сервиса с открытыми данными;
& обмен файлами по зашифрованному (TODO см. выше) каналу связи по принципу REST API;
& парсинг csv-реестров, запакованных в архивах.
\end{easylist}

С программной точки зрения модуль интеграции РИАС ЖКХ с ГИС ЖКХ представляет службу ОС Windows.
Общее количество значимых строк кода превышает 10 тысяч.
При разработки модуля интеграции использовались следующие технологии и библиотеки:
\begin{easylist}
& .NET Framework 4.5;
& язык программирования Visual C\# 6;
& КриптоПРО .NET;
& Json.NET;
& NLog.
\end{easylist}

Модуль разделён на несколько составных частей:
\begin{easylist}
& ядро обмена (AIS.HM.Integration.GIS.Core);
& служба windows (AIS.HM.Integration.GIS.Production);
& тестовый клиент интеграции (AIS.HM.Integration.GIS.Test);
& фронт для взаимодействия с внутренними системами (AIS.HM.Integration.GIS.WebService);
& графический интерфейс для администратора системы и поставщиков информации (AIS.HM.UI.GIS).
\end{easylist}
Каждый компонент программного обеспечения использует модель данных РИАС ЖКХ.
Здесь можно выделить основные сущности РИАС ЖКХ (Организация, Дом, Помещение) и специфичные для модуля (ГИС\_Запрос, ГИС\_Операция, ГИС\_ЛогПлатёжногоШлюза).
Полное описание используемых сущностей логической (концептуальной) модели данных представлено в таблице TODO:таблица.

TODO: таблица (см. выше).

\subsubsection{Механизм информационного обмена}

Основными принципами информационного обмена с ГИС ЖКХ являются:
\begin{easylist}
& поддержание целостности данных, размещённый в РИАС ЖКХ и ГИС ЖКХ;
& максимально быстрая автоматическая выгрузка данных в ГИС ЖКХ;
& интеграция данных из смежных систем, развёрнутых вкупе с РИАС ЖКХ, и дальнейшая передача их в ГИС ЖКХ;
& гарантированность раскрытия информации в ГИС ЖКХ.
\end{easylist}

Согласно реализации информационного взаимодействия с ГИС ЖКХ, описанному в п. TODO:ссылка, было принято весь процесс отправки сообщений в федеральную систему разделить на следующие этапы:
\begin{enumerate}
	\item Для операции обмена (например, выгрузка договоров или приём домов) определяется набор поставщиков информации, которые имеют право формировать такие запросы.
	\item Для каждого поставщика информации формируется запросы: определяется набор данных, подлежащих обмену, формируются первичные XML-представления запросов (untrusted xml request), формируются пакеты в случае пакетной отправки данных.
	\item В отдельном потоке для одиночных запросов и в текущем потоке для пакетных запросов XML-представления подписываются по ГОСТ (TODO: ссылка на ГОСТ) и отправляются в веб-сервисы ГИС ЖКХ. ГУИД ответа (AckResult.MessageGUID) записывается в хранилище данных РИАС ЖКХ.
	\item В отдельном потоке идёт опрос результатов обработки запроса. Если запрос был обработан, запускается необходимый обработчик, который зависит от операции обмена и веб-сервиса.
\end{enumerate}

TODO: дополнить.

\subsection{Модуль интеграции РИАС ЖКХ с <<АИС Город. Система начислений>>}

... как пример взаимодействия с внутренней системой.

Собственно, аналогичные комментарии.

\clearpage
\newpage