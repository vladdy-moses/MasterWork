\section{Описание предметной области}

\subsection{Описание ГИС ЖКХ России}

Государственная информационная система жилищно-коммунального хозяйства России --- программный комплекс, реализуемый в рамках 209 ФЗ (TODO: ссылка) и устанавливающий следующие цели:
\begin{easylist}
& увеличение прозрачности сферы ЖКХ страны;
& унификация электронных видов информации, связанных с жилищно-коммунальным хозяйством;
& объединение разнообразных федеральных информационных систем, связанных с ЖКХ;
& снижение расходов на информационное обеспечение участников сферы ЖКХ.
\end{easylist}

Для достижения поставленных целей перед ГИС ЖКХ поставлены следующие задачи:
\begin{easylist}
& закрепление на федеральном уровне обязанности раскрывать информацию в ГИС ЖКХ всеми участниками сферы ЖКХ;
& закрепление на федеральном уровне перечня, способов и сроков раскрытия информации в ГИС ЖКХ;
& реализация информационного взаимодействия с федеральными системами и сервисами электронного правительства России;
& реализация информационного взаимодействия с региональными, муниципальными, коммерческими и собственными информационными системами;
& возможность формирования отчётов, в том числе для аналитических исследований.
\end{easylist}

Согласно 209 ФЗ (TODO: ссылка) ГИС ЖКХ должна учитывать следующие принципы: (TODO: ссылка)
\begin{easylist}
& открытость, прозрачность и общедоступность информации;
& однократность размещения в системе информации;
& многократность использования информации, размещённой в системе;
& непрерывность и бесперебойность функционирования системы;
& полнота, достоверность, актуальность информации и своевременность её размещения в системе;
& бесплатность размещения в системе информации;
& использование единых форматов для информационного взаимодействия иных информационных систем с системой.
\end{easylist}

\subsubsection{Устройство ГИС ЖКХ}

ГИС ЖКХ разделяется на открытую и закрытую часть.
Снимок экрана открытой части портала ГИС ЖКХ представлен на рисунке~\ref{img:gis-open-screenshot}.

В закрытой части ГИС ЖКХ интерфейс зависит от функций организации в ГИС ЖКХ, а также от роли пользователя в организации.

В ГИС ЖКХ существует две роли: администратор организации и уполномоченный специалист.
Уполномоченный специалист может выполнять действия по раскрытию информации от имени организации согласно 209-ФЗ.
Администратор организации может устанавливать ограничения в правах уполномоченным специалистам, совершать в системе сделки от имени организации, передавать полномочия на раскрытие данных от имени организации другим компаниями или информационным системами.

Функций у одной организации может быть несколько.
Перечень основных функций организации, возможных в ГИС ЖКХ:
TODO: перечень функций организаций в ГИС ЖКХ.

Снимок экрана закрытой части портала ГИС ЖКХ для уполномоченного специалиста управляющей организации представлен на рисунке~\ref{img:gis-lk-uo-screenshot}.

\myFigure[t]{0.9}{gis-open-screenshot}{Снимок экрана открытой части портала ГИС ЖКХ}

\myFigure[t]{0.9}{gis-lk-uo-screenshot}{Снимок экрана закрытой части портала ГИС ЖКХ для уполномоченного специалиста управляющей организации}

С технической точки зрения ГИС ЖКХ можно разделить на веб-портал, REST API для портала, набор открытых данных и информационное взаимодействие с внешними системами по протоколам SOAP.

Обмен данными может быть построен в ручном или автоматическом режиме.
Для ручного режима предусмотрен ввод данных через веб-портал ГИС ЖКХ, а также при помощи Excel-шаблонов.
Для автоматического обмена существует информационное взаимодействие с внешними системами и набор открытых данных.

Необходимые данные для автоматического обмена находятся в разделе регламентов и инструкций.

Для проведения испытаний обмена с ГИС ЖКХ существуют стенды информационного тестирования СИТ-01 и СИТ-02.
Сам портал ГИС ЖКХ называется промышленным программного-аппаратным комплексом (ППАК).

Форматы информационного взаимодействия разделяются на текущие и перспективные.
Текущие форматы актуальны для ППАК ГИС ЖКХ и стенда СИТ-01.
Соответственно, перспективные форматы актуальны для стенда СИТ-02.

\subsection{Описание РИАС ЖКХ субъекта России}

Региональная аналитическая система жилищно-коммунального хозяйства субъекта Российской Федерации (РИАС ЖКХ) --- программный комплекс решений по автоматизации сферы ЖКХ.
РИАС ЖКХ разрабатывается с 2010 года компанией <<АИС Город>> (Ульяновск).

Отличительные особенности РИАС ЖКХ по сравнению с аналогичными разработками:
\begin{easylist}
& система может быть внедрена как в рамках субъекта России, так и не опираясь на конкретную территорию;
& тесная интеграция с федеральными и коммерческими информационными системами;
& наличие справочников ФИАС, адресного плана ГИС ЖКХ, ОКТМО, ОКЕИ, НСИ ГИС ЖКХ (TODO: аббревиатуры) и иных отраслевых справочников;
& модульность системы, благодаря чему достигается её низкая стоимость;
& реализация постановлений Правительства России (TODO: уточнить каких) в сфере жилищно-коммунального хозяйства;
& гибкая настройка развёртки системы под нужны заказчика;
& быстрая реакция технической поддержки.
\end{easylist}

Снимок экрана главной страницы РИАС ЖКХ для администратора системы представлен на рисунке~\ref{img:rias-main-screenshot}.

\myFigure[t]{0.9}{rias-main-screenshot}{Снимок экрана главной страницы РИАС ЖКХ для администратора системы}

\subsection{Проблема информационного взаимодействия}

% Здесь будет описание проблемы всего информационного взаимодействия в современном IT-мире.
% Много стандартов, много протоколов, постоянно растущая энтропия (технический долг)...

В современном IT-мире существует множество информационных систем и технологий, позволяющих хранить и обрабатывать данные.
Зачастую эти данные дублируются, а если данные занесены пользователем, данные могут дублироваться частично.
Соответственно, возникает множество конфликтных ситуаций.
Например, какую информационную систему в данном случае считать более эталонной или как избавиться от дублирования информации в различных системах.

Для разрешения последнего вопроса были разработаны методы взаимодействия между информационными системами, которые позволяют обмениваться информацией между системами в автоматическом или автоматизированном режимах.

Как и в любой сфере деятельности, в информационном взаимодействии инженеры постоянно стремились (и стремятся) всё формализовать и стандартизировать.
Появляются стандарты и протоколы, описывающие формат сообщений, требования к каналам передачи данных, защите информации.
Также формализуются шаблоны проектирования информационных систем для наименее затратной организации информационного обмена с иными системами.

Одни методы делают упор на скорость передачи данных, другие -- на контроль целостности и защищённость, третьи -- на лёгкость реализации механизмов обмена.
Всё это требуется учитывать при реализации информационного взаимодействия.

С точки зрения разработки региональной системы жилищно-коммунального хозяйства проблема информационного взаимодействия состоит в выборе методов обмена с муниципальными и собственными информационными системами и в реализации клиентов информационного обмена с федеральными информационными системами. 

% \subsection{Анализ аналогичных исследований}

% Надо будет найти вменяемые источники на тему изучения повадок интеграции нескольких информационных систем.

\subsection{Обоснование необходимости решения проблемы}

Грамотная организация информационного обмена крупной информационной системы, состоящей из множества модулей, позволит:
\begin{easylist}
& уменьшить количество вводимой пользователями информации;
& увеличить полезность системы на рынке ЖКХ;
& уменьшить несогласованность данных в различных информационных системах;
& добавить дополнительные точки роста системы;
& расширить отчётные данные (например, об использовании системы различными организациями).
\end{easylist}

\subsection{Определение информационного взаимодействия}

% TODO: добавить блок с определением?

\textbf{Информационное взаимодействие} --- процесс обмена информацией между источником и приёмником по каналам связи.
В более узком смысле информационное взаимодействие двух информационных систем (\textbf{межсистемный информационный обмен}) можно трактовать как процесс передачи информации между информационными системами при помощи телекоммуникационной сети Интернет.
В свою очередь \textbf{межмодульный информационный обмен} --- подвид информационного взаимодействия, производимый внутри информационной системы.

Несмотря на схожесть определений межсистемного и межмодульного обмена, они решают разные цели и характеризуются следующими отличиями:
\begin{easylist}
& межсистемный обмен надёжнее защищён нежели межмодульный;
& межсистемный обмен лучше документирован;
& межмодульный обмен намного быстрее межсистемного;
& межмодульный обмен может чаще обновляться, так как источник и приёмник информации -- сама информационная система;
& межмодульный обмен может использовать более узкий набор технологий.
\end{easylist}

\subsection{Виды современного информационного взаимодействия}

TODO: описать, где я их использую или почему не использую.

\subsubsection{Прямой обмен между БД}

Данный метод обычно используется при межмодульном взаимодействии.
Он заключается в том, что источник и приёмник используют одну и ту же базу данных либо связанные базы данных.
Например, СУБД Microsoft SQL Server позволяет связать несколько СУБД для доступа к базам данных не только той СУБД, где находится исходная база данных.

Преимуществами прямого доступа к базе данных является:
\begin{easylist}
& высокая скорость работы;
& отсутствие лишних издержек для построения взаимодействия.
\end{easylist}

Несмотря на преимущества, у данного метода есть и очевидные недостатки:
\begin{easylist}
& отсутствие какой-либо защиты данных от несанкционированного доступа к ним;
& <<привязка>> к определённой СУБД или технологии;
& сложность изменения форматов взаимодействия.
\end{easylist}

Прямой обмен между базами данных можно разделить на активный и пассивный.
При активном обмене источник данных сам передаёт данные в базу данных получателя.
При пассивном обмене происходит обратный процесс: получатель самостоятельно забирает данные из базы данных источника.

Также такой вид информационного взаимодействия можно разделить на автоматизированный и автоматический.
При автоматизированном обмене персонал должен инициировать обмен.
При автоматическом обмене участие персонала не требуется.

\subsubsection{Обмен реестрами}

Обмен структурированными файлами (реестрами) берёт своё начало очень давно: ещё со времени, когда многие информационные системы не были соединены между собой прямыми каналами связи (как сейчас при помощи сети Интернет).
Однако несмотря на тотальное объединение информационных систем в одну сеть, данный метод информационного обмена является одним из самых популярных.

Преимущества обмена реестрами:
\begin{easylist}
& простота реализации;
& информационные системы, разработанные давно, как правило, поддерживают обмен реестрами;
& организация защиты информации в данном методе обмена -- не обязанность этого метода;
\end{easylist}

Недостатки обмена реестрами:
\begin{easylist}
& принципиальная сложность передавать бинарные файлы;
& огромное количество кодировок могут исказить информацию;
& могут использоваться устаревшие технологии (например, dBase III).
\end{easylist}

В текущий момент набирает популярность аналог данного метода информационного обмена: предоставление шаблонизированных книг Microsoft Excel.
С одной стороны, с ними гораздо удобнее работать, нежели с файлами CSV или таблицами dBase, однако для этой работы требуется платное программное обеспечение.

\subsubsection{SOAP}

SOAP представляет собой легковесный протокол, предназначенный для обмена структурированными данными в децентрализованной информационной среде~\cite{soapSpec}.
Последняя актуальная версия протокола --- 1.2.
Для определения форматов сообщения протокол использует XML.

SOAP не предъявляет чётких требований к протоколу передачи данных, безопасности и гарантии доставки сообщений, однако позволяет использовать для этих целей расширения~\cite{soapSpec}.
Обычно для протокола передачи данных используется HTTPS или HTTP, но возможны и более экзотические варианты, например, SMTP.

Для описания схем информационного обмена используется XSD.
Для определения веб-служб используется язык WSDL.

Преимущества SOAP:
\begin{easylist}
& формат сообщений стандартизирован;
& может быть использован любой протокол прикладного уровня;
& легко обеспечить защиту сообщений при помощи подписи данных и запросов;
& существует множество реализаций серверов и клиентов для современных языков программирования;
& самодокументируемость протокола позволяет сократить объём артефактов, необходимых для организации информационного обмена.
\end{easylist}

Недостатки SOAP:
\begin{easylist}
& используется избыточный формат сообщений --- XML;
& необходима высокая квалификация аналитиков и разработчиков веб-сервиса, чтобы им было удобно пользоваться;
& протокол сложен в реализации на мобильных устройствах и некоторых настольных системах.
\end{easylist}

Протокол SOAP используется в основном для организации информационного взаимодействия между внешними системами, слабо связанными друг с другом.
Например, такой вид обмена может быть настроен между информационной системой государственного или регионального уровня с более мелкими ИС.
Для построения микросервисной архитектуры внутри программного комплекса или реализации обмена с мобильными платформами и фронтами веб-приложений рекомендуется использовать подход REST.

\subsubsection{REST}

Существует альтернативный протоколу SOAP механизм информационного обмена в WWW, распространённый в текущее время.

REST определяет набор архитектурных принципов, придерживаясь которых можно создавать веб-сервисы, основываясь лишь на том, как информация будет передаваться по протоколам HTTP (HTTPS) к клиентам, написанным на разных языках программирования~\cite{restBasics}.

Основные принципы REST-подхода к построению API следующие:
\begin{easylist}
& явное использование методов HTTP (GET, POST, PUT, DELETE);
& запросы к API не зависят друг от друга (отсутствует понятие состояния или сессии на стороне сервера);
& URL-адреса доступа к API структурированы;
& для описания данных используется XML, JSON (рекомендуется) или обе технологии вместе~\cite{restBasics}.
\end{easylist}

Преимущества подхода REST к построению информационного взаимодействия:
\begin{easylist}
& высокая полезная нагрузка сообщений из-за использования JSON;
& более лёгкая обработка данных при помощи современных технологий;
& из-за отсутствия на сервере сессий можно легко строить отказоустойчивые кластеры.
\end{easylist}

Недостатки REST:
\begin{easylist}
& нет единого принятого стандарта описания форматов сообщений;
& отсутствует единое описание защиты передаваемых данных;
& если веб-сервис большой, или если веб-сервисов много, легко можно запутаться в URL-адресах конечных точек.
\end{easylist}

Данный вид взаимодействия хорошо себя показал при обмене данными внутри веб-приложений между клиентским и серверным кодом.
Подход REST рекомендуется использовать при общении с мобильными приложениями.
Также такой принцип можно использовать при организации синхронизации между модулями программного комплекса.

\subsection{Правовые основы информационного обмена в ЖКХ}

Ниже перечислены основные законы и подзаконные акты Российской Федерации, согласно которым любые организации, относящиеся к сфере жилищно-коммунального хозяйства, должны вести отчётность в электронной форме:

\begin{enumerate}
	\item Федеральный закон от 21 июля 2014 года N 209-ФЗ <<О государственной информационной системе жилищно-коммунального хозяйства>>
	\item Приказ от 29.02.2016 года № 74/114/пр <<Об утверждении состава, сроков и периодичности размещения информации поставщиками информации в государственной информационной системе жилищно-коммунального хозяйства>>
	\item Приказ от 02.03.2016 года № 77/120/пр <<Об утверждении состава, порядка, сроков и периодичности размещения в государственной информационной системе жилищно-коммунального хозяйства информации о предоставлении субъектам Российской Федерации и муниципальным образованиям финансовой поддержки на проведение капитального ремонта многоквартирных домов, переселение граждан из аварийного жилищного фонда, модернизацию систем коммунальной инфраструктуры, а также о выполнении условий предоставления такой финансово>>
	\item Приказ от 28.01.2016 года № 18/34/пр <<Об утверждении состава, порядка, способов, сроков и периодичности размещения в государственной информационной системе жилищно-коммунального хозяйства информации о количестве зарегистрированных в жилых помещениях по месту пребывания и по месту жительства граждан>>
	\item Приказ от 28.12.2015 года № 589/944/пр <<Об утверждении Порядка и способов размещения информации, ведения реестров в государственной информационной системе жилищно-коммунального хозяйства, доступа к системе и к информации, размещённой в ней>>
	\item Постановление Правительства РФ от 23.09.2010 года № 731 <<Об утверждении стандарта раскрытия информации организациями, осуществляющими деятельность в сфере управления многоквартирными домами>>
	\item Приказ от 1 декабря 2016 года № 871/пр <<Об утверждении форм мониторинга и отчётности реализации субъектами Российской Федерации региональных программ капитального ремонта общего имущества в многоквартирных домах и признании утратившими силу отдельных Приказов Минстроя России>>
\end{enumerate}

TODO: добавить, как законы помогают или мешают работать.

\clearpage
\newpage