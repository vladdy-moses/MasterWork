\section{Описание проблемы исследования}

\subsection{Проблема информационного взаимодействия}

% Здесь будет описание проблемы всего информационного взаимодействия в современном IT-мире.
% Много стандартов, много протоколов, постоянно растущая энтропия (технический долг)...

В современном IT-мире существует множество информационных систем и технологий, позволяющих хранить и обрабатывать данные.
Зачастую эти данные дублируются, а если данные занесены пользователем, данные могут дублироваться частично.
Соответственно, возникает множество конфликтных ситуаций.
Например, какую информационную систему в данном случае считать более эталонной или как избавиться от дублирования информации в различных системах.

Для разрешения последнего вопроса были разработаны методы взаимодействия между информационными системами, которые позволяют обмениваться информацией между системами в автоматическом или автоматизированном режимах.

Как и в любой сфере деятельности, в информационном взаимодействии инженеры постоянно стремились (и стремятся) всё формализовать и стандартизировать.
Появляются стандарты и протоколы, описывающие формат сообщений, требования к каналам передачи данных, защите информации.
Также формализуются шаблоны проектирования информационных систем для наименее затратной организации информационного обмена с иными системами.

Одни методы делают упор на скорость передачи данных, другие -- на контроль целостности и защищённость, третьи -- на лёгкость реализации механизмов обмена.
Всё это требуется учитывать при реализации информационного взаимодействия.

С точки зрения разработки региональной системы жилищно-коммунального хозяйства проблема информационного взаимодействия состоит в выборе методов обмена с муниципальными и собственными информационными системами и в реализации клиентов информационного обмена с федеральными информационными системами. 

% \subsection{Анализ аналогичных исследований}

% Надо будет найти вменяемые источники на тему изучения повадок интеграции нескольких информационных систем.

\subsection{Обоснование необходимости решения проблемы}

Грамотная организация информационного обмена крупной информационной системы, состоящей из множества модулей, позволит:
\begin{enumerate}
	\item уменьшить количество вводимой пользователями информации;
	\item увеличить полезность системы на рынке ЖКХ;
	\item уменьшить несогласованность данных в различных информационных системах;
	\item добавить дополнительные точки роста системы;
	\item расширить отчётные данные (например, об использовании системы различными организациями).
\end{enumerate}

\subsection{Определение информационного взаимодействия}

% TODO: добавить блок с определением?

\textbf{Информационное взаимодействие} --- процесс обмена информацией между источником и приёмником по каналам связи.
В более узком смысле информационное взаимодействие двух информационных систем (\textbf{межсистемный информационный обмен}) можно трактовать как процесс передачи информации между информационными системами при помощи телекоммуникационной сети Интернет.
В свою очередь \textbf{межмодульный информационный обмен} --- подвид информационного взаимодействия, производимый внутри информационной системы.

Несмотря на схожесть определений межсистемного и межмодульного обмена, они решают разные цели и характеризуются следующими отличиями:
\begin{easylist}
& межсистемный обмен надёжнее защищён нежели межмодульный;
& межсистемный обмен лучше документирован;
& межмодульный обмен намного быстрее межсистемного;
& межмодульный обмен может чаще обновляться, так как источник и приёмник информации -- сама информационная система;
& межмодульный обмен может использовать более узкий набор технологий.
\end{easylist}

\subsection{Виды современного информационного взаимодействия}

\subsubsection{Прямой доступ к БД}

Данный метод обычно используется при межмодульном взаимодействии.
Он заключается в том, что источник и приёмник используют одну и ту же базу данных либо связанные базы данных.
Например, СУБД Microsoft SQL Server позволяет связать несколько СУБД для доступа к базам данных не только той СУБД, где находится исходная база данных.

Преимуществами прямого доступа к базе данных является:
\begin{easylist}
& высокая скорость работы;
& отсутствие лишних издержек для построения взаимодействия.
\end{easylist}

Несмотря на преимущества, у данного метода есть и очевидные недостатки:
\begin{easylist}
& отсутствие какой-либо защиты данных от несанкционированного доступа к ним;
& <<привязка>> к определённой СУБД или технологии;
& сложность изменения форматов взаимодействия.
\end{easylist}

\subsubsection{SOAP}

TODO: описать, что такое SOAP и что же такого прекрасного в этом протоколе.

Преимущества SOAP:
\begin{easylist}
& формат сообщений стандартизирован;
& может быть использован любой протокол прикладного уровня;
& легко обеспечить защиту сообщений при помощи подписи данных и запросов;
\end{easylist}

Недостатки SOAP:
\begin{easylist}
& используется избыточный формат сообщений -- XML;
& протокол сложен в реализации на мобильных устройствах и некоторых настольных системах.
\end{easylist}

\subsubsection{REST}

TODO: описать, что такое REST-подход к архитектуре взаимодействия.

TODO: описать про то, что определение REST очень неоднозначно.

Преимущества REST:
\begin{easylist}
& высокая полезная нагрузка сообщений из-за использования JSON;
& более лёгкая обработка данных при помощи современных технологий;
\end{easylist}

Недостатки REST:
\begin{easylist}
& нет единого принятого стандарта описания форматов сообщений;
& отсутствует единое описание защиты передаваемых данных.
\end{easylist}

\subsection{Правовые основы информационного обмена в ЖКХ}

Ниже перечислены основные законы и подзаконные акты Российской Федерации, согласно которым любые организации, относящиеся к сфере жилищно-коммунального хозяйства, должны вести отчётность в электронной форме:

\begin{enumerate}
	\item Федеральный закон от 21 июля 2014 года N 209-ФЗ <<О государственной информационной системе жилищно-коммунального хозяйства>>
	\item Приказ от 29.02.2016 года № 74/114/пр <<Об утверждении состава, сроков и периодичности размещения информации поставщиками информации в государственной информационной системе жилищно-коммунального хозяйства>>
	\item Приказ от 02.03.2016 года № 77/120/пр <<Об утверждении состава, порядка, сроков и периодичности размещения в государственной информационной системе жилищно-коммунального хозяйства информации о предоставлении субъектам Российской Федерации и муниципальным образованиям финансовой поддержки на проведение капитального ремонта многоквартирных домов, переселение граждан из аварийного жилищного фонда, модернизацию систем коммунальной инфраструктуры, а также о выполнении условий предоставления такой финансово>>
	\item Приказ от 28.01.2016 года № 18/34/пр <<Об утверждении состава, порядка, способов, сроков и периодичности размещения в государственной информационной системе жилищно-коммунального хозяйства информации о количестве зарегистрированных в жилых помещениях по месту пребывания и по месту жительства граждан>>
	\item Приказ от 28.12.2015 года № 589/944/пр <<Об утверждении Порядка и способов размещения информации, ведения реестров в государственной информационной системе жилищно-коммунального хозяйства, доступа к системе и к информации, размещённой в ней>>
	\item Постановление Правительства РФ от 23.09.2010 года № 731 <<Об утверждении стандарта раскрытия информации организациями, осуществляющими деятельность в сфере управления многоквартирными домами>>
	\item Приказ от 1 декабря 2016 года № 871/пр <<Об утверждении форм мониторинга и отчётности реализации субъектами Российской Федерации региональных программ капитального ремонта общего имущества в многоквартирных домах и признании утратившими силу отдельных Приказов Минстроя России>>
\end{enumerate}

\clearpage
\newpage