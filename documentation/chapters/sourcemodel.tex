\section{Модели, методы и алгоритмы исследования}

\subsection{Модель информационного взаимодействия}

Для формального определения информационного взаимодействия необходимо определить его элементы.

Прежде всего, в обмене информацией участвует источник данных $S$ и приёмник $D$.
В реальной жизни ими являются программного-аппаратные комплексы информационных систем, однако в случае моделирования и источник, и приёмник можно описать конечным набором некоторых характеристик.
Представим $A_S$ как множество характеристик источника и $A_D$ --- множество характеристик приёмника.
Тогда $A_S \cap A_D = A_{SD}$ --- множество совпадающих характеристик и источника, и приёмника данных.
Для упрощения понимания сути информационного взаимодействия представим источник и приёмник данных через столбцы их характеристик:
$$S = \begin{bmatrix}
s_{1} \\
\vdots \\
s_{n} \\         
\vdots \\
s_{k}
\end{bmatrix},
D = \begin{bmatrix}
d_{1} \\
\vdots \\
d_{n} \\         
\vdots \\
d_{m}
\end{bmatrix},$$
где $n$ -- мощность множества $A_{SD}$ ($n = |A_{SD}|$), $k = |A_S|$, $m = |A_D|$, $s \subset A_S$, $d \subset A_D$.
Примеры характеристик источника и приёмника данных описаны в п.~\ref{sec:modelSDChars}.

Данные передаются по каналу связи $C$.
Им может являться телекоммуникационная сеть Интернет, локальная вычислительная сети предприятия и даже ручная передача данных от источника к приёмнику.
Канал связи также обладает некоторым набором характеристик $A_C$, через которые можно его описать.
Таким образом,
$$
C = \begin{bmatrix}
c_{1} \\
\vdots \\
c_{m}
\end{bmatrix},
$$
где $m = |A_C|$, $c \subset A_C$.
Примеры характеристик канала связи описаны в п.~\ref{sec:modelCChars}.

Данные от источника к приёмнику передаются по определённым правилам, закреплённым в протоколе информационного обмена $P$.
С точки зрения моделирования процесса информационного взаимодействия нас не интересуют отдельные правила передачи данных, закреплённые в протоколе.
Таким образом, протокол $P$ также можно описать при помощи столбца характеристик $A_P$:
$$
P = \begin{bmatrix}
p_{1} \\
\vdots \\
p_{j}
\end{bmatrix},
$$
где $j = |A_P|$, $p \subset A_P$.
Примеры характеристик протокола описаны в п.~\ref{sec:modelPChars}.

Информационный обмен невозможен без описания самих данных, передаваемых в ходе взаимодействия.
Для упрощения назовём такие данные сообщениями.
Пусть множество $A_M$ -- множество характеристик передаваемых сообщений.
Тогда сообщения $M$ можно описать через набор их характеристик:
$$
M = \begin{bmatrix}
m_{1} \\
\vdots \\
m_{q}
\end{bmatrix},
$$
где $q = |A_M|$, $m \subset A_M$.
Примеры характеристик данных (сообщений) описаны в п.~\ref{sec:modelMChars}.

Таким образом, информационное взаимодействие $E$ можно определить следующим образом:
$$
E = <S,D,C,P,M>,
$$
где $E$ -- информационное взаимодействие, $S$ -- источник данных, $D$ -- приёмник данных, $C$ -- канал связи, $P$ -- протокол передачи данных, $M$ -- передаваемое сообщение.
Визуально описанные выше компоненты можно представить в виде рисунка~\ref{img:model-general}.

\myFigure[t]{0.9}{model-general}{Упрощённая схема информационного взаимодействия}

TODO: связи компонентов информационного взаимодействия.

\subsection{Характеристики информационного взаимодействия}

TODO: вступительное описание

\subsubsection{Характеристики источника и приёмника данных}
\label{sec:modelSDChars}

Общими характеристиками источника и приёмника данных могут являться:
\begin{easylist}
& скорость обработки данных;
& максимальное число потоков обработки данных;
& вид и величина деградации скорости обработки данных при увеличении объёма;
& вид и величина деградации скорости обработки данных при увеличении потоков.
\end{easylist}

TODO: ещё

\subsubsection{Характеристики канала связи}
\label{sec:modelCChars}

Характеристиками канала связи могут являться:
\begin{easylist}
& максимальная пропускная способность;
& процент ошибок при передачи данных;
& возможность определения и/или исправления ошибок;
& предельная дальность передачи информации;
& временная задержка при передаче.
\end{easylist}

TODO: подробное описание.

\subsubsection{Характеристики протокола передачи данных}
\label{sec:modelPChars}

Характеристиками протокола передачи данных могут являться:
\begin{easylist}
& коэффициент избыточности сериализованных данных;
& сложность сериализации и десериализации;
& возможность шифрования;
& возможность валидации.
\end{easylist}

TODO: ещё + подробное описание.

\subsubsection{Характеристики данных (сообщений)}
\label{sec:modelMChars}

Характеристиками сообщений могут являться:
\begin{easylist}
& разнородность;
& структурированность;
& связность;
& средний объём сообщения;
& средний объём единицы информации в сообщении.
\end{easylist}

TODO: ещё + описать

\subsection{Основные алгоритмы построения информационного взаимодействия}

\subsubsection{Организация прямого обмена}

Перед реализацией прямого обмена необходимо решить, какие СУБД будут обмениваться информацией, так как не все системы могут взаимодействовать друг с другом.
Например, СУБД Oracle и SQL Server могут выполнять запросы друг к другу, а вот SQL Server и MySQL -- нет.

Затем необходимо выбрать вариант обмена: активный или пассивный и автоматический или автоматизированный.
Об этом более подробно описано в п.~\ref{descrDB}.

Затем необходимо источнику и приёмнику данных договориться о форматах информационного взаимодействия.
Это может быть описание физической модели данных, набора хранимых процедур (с описанием логики их работы но без конкретной релизации) или логической модели данных, если последнее допустимо и необходимо.
В случае реляционных СУБД обычно создаётся отдельный набор таблиц для прямого обмена как у источника, так у приёмника.
Также допускается создание представлений для обмена.

Дополнительно на этапе проектирования необходимо согласовать правила занесения данных в физическую модель данных прямого обмена.
Этот шаг позволяет значительно сократить время на интеграционные испытания и решение конфликтных ситуаций, которые часто возникают при информационном обмене, спроектированном без участия всех заинтересованных сторон.

После согласования разработчики информационных систем приступают к реализации моделей данных и механизмов их заполнения, а также обмена данными между моделями.
На стороне источника данных обычно создаются:
\begin{easylist}
& физическая модель данных;
& механизм заполнения и актуализации модели прямого обмена;
& механизм проверки целостности данных в модели прямого обмена.
\end{easylist}
На стороне приёмника данных обычно создаются те же элементы совместно с:
\begin{easylist}
& механизм контроля связи до источника данных;
& механизм обработки данных из модели прямого обмена;
& механизм проверки корректности данных, занесённый в систему из модели прямого обмена.
\end{easylist}
Дополнительно рекомендуется реализовать механизм кросс-контроля, когда источник данных может проверить собственные данные в системе-приёмнике, однако для этого необходимо либо сильно усложнять прямой обмен, либо использовать другие методы прямого обмена.

После реализации происходит процесс интеграционного тестирования, когда разработчики источника и приёмника совместными усилиями проверяют обмен и все связанные с ним механизмы в своих системах.
Особое внимание на данном этапе стоит уделять корректности данных в модели прямого обмена между источником и приёмником.
К примеру, необходимо удостовериться, что все поля из физической модели источника корректно разложились по соответствующим полям физической модели прямого обмена приёмника данных.
Любые несоответствия должны быть разрешены.
Например, одна система отсутствием данных считает пустую строку, а другая -- специальное значение NULL.
Такой случай необходимо согласовать отдельно и задокументировать.

После проведения интеграционных испытаний и необходимых доработок источника и приёмника данных механизм переходит в промышленную эксплуатацию.

\subsubsection{Обмен реестрами}

TODO

\subsubsection{Построение веб-сервисов SOAP и REST API}

TODO

\clearpage
\newpage