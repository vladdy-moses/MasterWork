\section{Модели, методы и алгоритмы исследования}

\subsection{Модель информационного взаимодействия}

Для формального определения информационного взаимодействия необходимо определить его элементы.

Прежде всего, в обмене информацией участвует источник данных $S$ и приёмник $D$.
В реальной жизни ими являются программного-аппаратные комплексы информационных систем, однако в случае моделирования и источник, и приёмник можно описать конечным набором некоторых характеристик.
Представим $A_S$ как множество характеристик источника и $A_D$ --- множество характеристик приёмника.
Тогда $A_S \cap A_D = A_{SD}$ --- множество совпадающих характеристик и источника, и приёмника данных.
Для упрощения понимания сути информационного взаимодействия представим источник и приёмник данных через столбцы их характеристик:
$$S = \begin{bmatrix}
s_{1} \\
\vdots \\
s_{n} \\         
\vdots \\
s_{k}
\end{bmatrix},
D = \begin{bmatrix}
d_{1} \\
\vdots \\
d_{n} \\         
\vdots \\
d_{m}
\end{bmatrix},$$
где $n$ -- мощность множества $A_{SD}$ ($n = |A_{SD}|$), $k = |A_S|$, $m = |A_D|$, $s \subset A_S$, $d \subset A_D$.
Примеры характеристик источника и приёмника данных описаны в п.~\ref{sec:modelSDChars}.

Данные передаются по каналу связи $C$.
Им может являться телекоммуникационная сеть Интернет, локальная вычислительная сети предприятия и даже ручная передача данных от источника к приёмнику.
Канал связи также обладает некоторым набором характеристик $A_C$, через которые можно его описать.
Таким образом,
$$
C = \begin{bmatrix}
c_{1} \\
\vdots \\
c_{m}
\end{bmatrix},
$$
где $m = |A_C|$, $c \subset A_C$.
Примеры характеристик канала связи описаны в п.~\ref{sec:modelCChars}.

Данные от источника к приёмнику передаются по определённым правилам, закреплённым в протоколе информационного обмена $P$.
С точки зрения моделирования процесса информационного взаимодействия нас не интересуют отдельные правила передачи данных, закреплённые в протоколе.
Таким образом, протокол $P$ также можно описать при помощи столбца характеристик $A_P$:
$$
P = \begin{bmatrix}
p_{1} \\
\vdots \\
p_{j}
\end{bmatrix},
$$
где $j = |A_P|$, $p \subset A_P$.
Примеры характеристик протокола описаны в п.~\ref{sec:modelPChars}.

Информационный обмен невозможен без описания самих данных, передаваемых в ходе взаимодействия.
Для упрощения назовём такие данные сообщениями.
Пусть множество $A_M$ -- множество характеристик передаваемых сообщений.
Тогда сообщения $M$ можно описать через набор их характеристик:
$$
M = \begin{bmatrix}
m_{1} \\
\vdots \\
m_{q}
\end{bmatrix},
$$
где $q = |A_M|$, $m \subset A_M$.
Примеры характеристик данных (сообщений) описаны в п.~\ref{sec:modelMChars}.

Таким образом, информационное взаимодействие $E$ можно определить следующим образом:
$$
E = <S,D,C,P,M>,
$$
где $E$ -- информационное взаимодействие, $S$ -- источник данных, $D$ -- приёмник данных, $C$ -- канал связи, $P$ -- протокол передачи данных, $M$ -- передаваемое сообщение.
Визуально описанные выше компоненты можно представить в виде рисунка~\ref{img:model-general}.

\myFigure[t]{0.9}{model-general}{Упрощённая схема информационного взаимодействия}

TODO: связи компонентов информационного взаимодействия.

\subsection{Характеристики информационного взаимодействия}

TODO: вступительное описание

\subsubsection{Характеристики источника и приёмника данных}
\label{sec:modelSDChars}

Общими характеристиками источника и приёмника данных могут являться:
\begin{easylist}
& скорость обработки данных;
& максимальное число потоков обработки данных;
& вид и величина деградации скорости обработки данных при увеличении объёма;
& вид и величина деградации скорости обработки данных при увеличении потоков.
\end{easylist}

TODO: ещё

\subsubsection{Характеристики канала связи}
\label{sec:modelCChars}

Характеристиками канала связи могут являться:
\begin{easylist}
& максимальная пропускная способность;
& процент ошибок при передачи данных;
& возможность определения и/или исправления ошибок;
& предельная дальность передачи информации;
& временная задержка при передаче.
\end{easylist}

TODO: подробное описание.

\subsubsection{Характеристики протокола передачи данных}
\label{sec:modelPChars}

Характеристиками протокола передачи данных могут являться:
\begin{easylist}
& коэффициент избыточности сериализованных данных;
& сложность сериализации и десериализации;
& возможность шифрования;
& возможность валидации.
\end{easylist}

TODO: ещё + подробное описание.

\subsubsection{Характеристики данных (сообщений)}
\label{sec:modelMChars}

Характеристиками сообщений могут являться:
\begin{easylist}
& разнородность;
& структурированность;
& связность;
& средний объём сообщения;
& средний объём единицы информации в сообщении.
\end{easylist}

TODO: ещё + описать

\subsection{Основные алгоритмы построения информационного взаимодействия}

\subsubsection{Организация прямого обмена}

TODO

\subsubsection{Обмен реестрами}

TODO

\subsubsection{Построение веб-сервисов SOAP и REST API}

TODO

\clearpage
\newpage