\section{Модели, методы и алгоритмы исследования}

\subsection{Определение информационного взаимодействия}

Тут даём определение вообще тому, что такое информационное взаимодействие.
Также вспоминает о понятии <<Информационная система>>.
По возможности даём определение терминам <<внутренняя система>> и <<внешняя система>>.

\subsection{Модель информационного взаимодействия}

Здесь будет формальное описание информационного взаимодействия.
Скорее всего, оперировать будем терминами кибернетики, в частности теории информации и её энтропийного подхода.

\subsection{Характеристики информационного взаимодействия}

Здесь будут описаны основные характеристики, которые нам важны при формировании модели информационного взаимодействия.
Возможно, на все эти характеристики мы будем накладывать ряд экспериментов.

\subsubsection{Устойчивость}

TBD

\subsubsection{Скорость}

TBD

\subsubsection{... ещё что-нибудь}

TBD

\subsection{Виды современного информационного взаимодействия}

Тут будет описание того, как сейчас обмениваются современные инфомационные системы.
Надо ли?
Думаю, надо.

\clearpage
\newpage