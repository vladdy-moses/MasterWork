\section{Описание экспериментов}

\subsection{Нагрузочное тестирование веб-сервиса платежей ГИС ЖКХ}
\label{expGis}
\subsubsection*{Цели и задачи эксперимента}

Целью данного вычислительного эксперимента является проверка корректности работы веб-сервиса регистрации платежей (фактов оплат) ГИС ЖКХ в момент пиковых нагрузок.

Данная цель достигается при помощи следующих задач:
\begin{easylist}
& внедрение механизма мониторинга времени обработки сообщений в ГИС ЖКХ для платёжного веб-сервиса;
& получение от заказчиков контрольных значений по пиковым нагрузкам их биллинговых систем;
& получение платежей (фактов оплат) в РИАС ЖКХ и их последующая отправка в ГИС ЖКХ в момент пиковой нагрузки;
& сбор и обработка полученных временных интервалов обработки пакетов фактов оплат и аппроксимация результатов для большего числа платежей.
\end{easylist}

\subsubsection*{Исходные данные}

Было получено разрешение от одного из заказчиков провести данный эксперимент на промышленных стендах РИАС ЖКХ и ГИС ЖКХ в рамках нагрузочных испытаний РИАС ЖКХ.

Время и продолжительность пика оплат за ЖКХ оказалось с 10:00 до 14:00.
В это время происходит свыше половины фактов оплат.
Каждую минуту от заказчика приходит в среднем около 120 платежей.
Наибольший поток платежей за 2017 год был зарегистрирован 27 февраля 2017 года в 14:03 и составил 533 платежа в минуту.

По требованию приказа №~74/114/пр~\cite{pr74114} и технического задания на РИАС ЖКХ время отправки факта оплаты от его совершения до конца его регистрации в ГИС ЖКХ должно составлять 2 часа.

Необходимо узнать, будет ли превышен указанный интервал в конце пиковой нагрузки.

\subsubsection*{План эксперимента}

\begin{enumerate}
	\item Сбор данных о пиковых нагрузках. На этом этапе необходимо выяснить у заказчика (или разработчика его биллинговой системы), в какие промежутки времени происходит самый большой поток оплат за услуги ЖКХ.
	\item Разработка и внедрение механизма мониторинга времени обработки запроса в ГИС ЖКХ.
	\item Получение фактов оплат от заказчика при пиковой нагрузке.
	\item Регистрация фактов оплат в ГИС ЖКХ. На данном шаге следует отметить, что факты оплат в ГИС ЖКХ регистрируются пакетами от 1 до 1000 штук. Здесь следует отправить пакеты по 1000 платежей (пик), а также и более мелкие пакеты (полупик) для выявления зависимости числа времени обработки пакета от числа платежей в нём.
	\item Моделирование и аппроксимация полученных задержек регистрации платежей на весь период пика. На данном шаге вычисляется максимальная задержка платежа под конец пиковой нагрузки.
	\item Составление выводов о корректности работы веб-сервиса ГИС ЖКХ.
\end{enumerate}

\subsubsection*{Выполнение эксперимента}

Для проведения нагрузочного тестирования на промышленном стенде заказчика в ночное время был развёрнут тестовый эксземпляр веб-сервиса по приёму платежей (фактов оплат).

Затем на платформе Microsoft Visual Studio был разработан нагрузочный тест, который параллельно в заданное число потоков отправлял запросы на регистрацию платежей.
Конфигурация нагрузочного теста представлена в таблице~\ref{tab:exp-gisapi-config}.

\begin{myTable}
\begin{longtable}[h]{|p{0.3\textwidth}|p{0.6\textwidth}|}
	\caption{\label{tab:exp-gisapi-config}Конфигурация нагрузочного теста для проведения испытаний платёжного шлюза} \\
	\hline
		\thead{Параметр} &
		\thead{Значение} \\
	\hline
		\theadnum{1} & \theadnum{2} \\
	\hline \endfirsthead
	\hline
		\theadnum{1} & \theadnum{2} \\
	\hline \endhead
		ПО веб-сервиса & IIS Express \\ \hline
		Количество одновременных клиентов API & 10 \\ \hline
		Метод API & SendOrderInfoAgent (регистрация факта оплаты платёжным агентом) \\ \hline
		Время тестирования & 5 минут \\ \hline
		Конфигурация веб-сервиса & Release \\ \hline
\end{longtable}
\end{myTable}

Снимок экрана с загрузкой процессора и оперативной памяти представлен на рисунке~\ref{img:exp-gisapi-test}.

Иные показатели результата нагрузочного теста представлены на рисунке~\ref{img:exp-gisapi-test-other}.

\myFigure[t]{0.9}{exp-gisapi-test}{Снимок экрана с загрузкой процессора и оперативной памяти. ЦП -- красным, ОЗУ -- зелёным}

\myFigure[t]{0.9}{exp-gisapi-result}{Иные показатели результата нагрузочного теста}

\subsubsection*{Ожидаемые и полученные результаты}

Ожидается, что веб-сервис будет обрабатывать более 1000 платежей в минуту.
Процент ошибок должен быть менее 0.3\%.

По результатам нагрузочного теста платёжный шлюз РИАС ЖКХ смог обработать за 5 минут 21\,830 платежей, что больше ожидаемого результата в 4 раза.
Ошибок было зарегистрировано 0, что меньше ожидаемого результата.

\subsubsection*{Выводы}

Платёжный шлюз РИАС ЖКХ, развёрнутый на площадке заказчика, успешно справляется с пиковыми нагрузками и имеет запас по скорости регистрации и обработки фактов оплат и не требует проведения дополнительной оптимизации программного обеспечения и улучшения аппаратного обеспечения серверов.

\subsection{Нагрузочное тестирование веб-сервиса платежей РИАС ЖКХ}

\subsubsection*{Цели и задачи эксперимента}

Данный эксперимент связан с экспериментом по нагрузочному тестированию веб-сервиса платежей ГИС ЖКХ, описанном в п.~\ref{expGis}.

Целью данного эксперимента является проверка корректности веб-сервиса РИАС ЖКХ по регистрации фактов оплат от биллинговых систем заказчиков (т.н. <<платёжный шлюз>>) и подсистемы обработки платежей РИАС ЖКХ для их дальнейшей отправки в ГИС ЖКХ.

Задачи эксперимента:
\begin{easylist}
& сбор сведений о предельной скорости регистрации фактов оплат в РИАС ЖКХ;
& сбор сведений о скорости обработки фактов оплат для их последующей отправки в ГИС ЖКХ;
& сопоставление полученных данных со скоростью обработки платежей в ГИС ЖКХ (более подробно описано в п.~\ref{expGis}).
\end{easylist}

\subsubsection*{Исходные данные}

Существует специально развёрнутая площадка РИАС ЖКХ для проведения нагрузочного тестирования на промышленных мощностях заказчика.

Также из отчёта о нагрузочном тестировании биллинговой системы заказчика известно, что данная система способна регистрировать до ХХХХХХ фактов оплат в час (ХХХХ в минуту).
Из п.~\ref{expGis} известно, что ГИС ЖКХ может обрабатывать до 4\,366 фактов оплат в минуту.

\subsubsection*{План эксперимента}

\begin{enumerate}
	\item Согласование с заказчиком времени и продолжительности проведения нагрузочного тестирования.
	\item Настройка РИАС ЖКХ для проведения нагрузочного тестирования веб-сервиса платежей.
	\item Разработка программного обеспечения для проверки времени ответа веб-сервиса платежей РИАС ЖКХ.
	\item Сбор статистики по времени ответа веб-сервиса платежей РИАС ЖКХ.
	\item Сбор статистики по времени обработки фактов оплат для их отправки в ГИС ЖКХ.
	\item Сопоставление полученных данных с исходными.
	\item Составление выводов.
\end{enumerate}

\subsubsection*{Выполнение эксперимента}

TODO

\subsubsection*{Ожидаемые и полученные результаты}

TODO

\subsubsection*{Выводы}

TODO

\subsection{Сравнение скорости передачи данных различными методами обмена}

\subsubsection*{Цели и задачи эксперимента}

TODO

\subsubsection*{Исходные данные}

TODO

\subsubsection*{План эксперимента}

TODO

\subsubsection*{Выполнение эксперимента}

TODO

\subsubsection*{Ожидаемые и полученные результаты}

TODO

\subsubsection*{Выводы}

TODO

\clearpage
\newpage