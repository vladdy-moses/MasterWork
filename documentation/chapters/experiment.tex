\section{Описание экспериментов}

\subsection{Нагрузочное тестирование веб-сервиса платежей ГИС ЖКХ}
\label{expGis}
\subsubsection*{Цели и задачи эксперимента}

Целью данного вычислительного эксперимента является проверка корректности работы веб-сервиса регистрации платежей (фактов оплат) ГИС ЖКХ в момент пиковых нагрузок.

Данная цель достигается при помощи следующих задач:
\begin{easylist}
& внедрение механизма мониторинга времени обработки сообщений в ГИС ЖКХ для платёжного веб-сервиса;
& получение от заказчиков контрольных значений по пиковым нагрузкам их биллинговых систем;
& получение платежей (фактов оплат) в РИАС ЖКХ и их последующая отправка в ГИС ЖКХ в момент пиковой нагрузки;
& сбор и обработка полученных временных интервалов обработки пакетов фактов оплат и аппроксимация результатов для большего числа платежей.
\end{easylist}

\subsubsection*{Исходные данные}

Было получено разрешение от одного из заказчиков провести данный эксперимент на промышленных стендах РИАС ЖКХ и ГИС ЖКХ в рамках нагрузочных испытаний РИАС ЖКХ.

Время и продолжительность пика оплат за ЖКХ оказалось с ХХ:ХХ до ХХ:ХХ (TODO часов) и с ХХ:ХХ до ХХ:ХХ (TODO часов).
В это время каждую минуту у заказчика происходит около ХХХХ фактов оплат.

По требованию TODO приказа (TODO: ссылка) и технического задания на РИАС ЖКХ время отправки факта оплаты от его совершения до конца его регистрации в ГИС ЖКХ должно составлять 2 часа.

Необходимо узнать, будет ли превышен указанный интервал в конце пиковой нагрузки.

\subsubsection*{План эксперимента}

\begin{enumerate}
	\item Сбор данных о пиковых нагрузках. На этом этапе необходимо выяснить у заказчика (или разработчика его биллинговой системы), в какие промежутки времени происходит самый большой поток оплат за услуги ЖКХ.
	\item Разработка и внедрение механизма мониторинга времени обработки запроса в ГИС ЖКХ.
	\item Получение фактов оплат от заказчика при пиковой нагрузке.
	\item Регистрация фактов оплат в ГИС ЖКХ. На данном шаге следует отметить, что факты оплат в ГИС ЖКХ регистрируются пакетами от 1 до 1000 штук. Здесь следует отправить пакеты по 1000 платежей (пик), а также и более мелкие пакеты (полупик) для выявления зависимости числа времени обработки пакета от числа платежей в нём.
	\item Моделирование и аппроксимация полученных задержек регистрации платежей на весь период пика. На данном шаге вычисляется максимальная задержка платежа под конец пиковой нагрузки.
	\item Составление выводов о корректности работы веб-сервиса ГИС ЖКХ.
\end{enumerate}

\subsubsection*{Выполнение эксперимента}

TODO

\subsubsection*{Ожидаемые и полученные результаты}

TODO

\subsubsection*{Выводы}

TODO

\subsection{Нагрузочное тестирование веб-сервиса платежей РИАС ЖКХ}

\subsubsection*{Цели и задачи эксперимента}

Данный эксперимент связан с экспериментом по нагрузочному тестированию веб-сервиса платежей ГИС ЖКХ, описанном в п.~\ref{expGis}.

Целью данного эксперимента является проверка корректности веб-сервиса РИАС ЖКХ по регистрации фактов оплат от биллинговых систем заказчиков (т.н. <<платёжный шлюз>>) и подсистемы обработки платежей РИАС ЖКХ для их дальнейшей отправки в ГИС ЖКХ.

Задачи эксперимента:
\begin{easylist}
& сбор сведений о предельной скорости регистрации фактов оплат в РИАС ЖКХ;
& сбор сведений о скорости обработки фактов оплат для их последующей отправки в ГИС ЖКХ;
& сопоставление полученных данных со скоростью обработки платежей в ГИС ЖКХ (более подробно описано в п.~\ref{expGis}).
\end{easylist}

\subsubsection*{Исходные данные}

Существует специально развёрнутая площадка РИАС ЖКХ для проведения нагрузочного тестирования на промышленных мощностях заказчика.

Также из отчёта о нагрузочном тестировании биллинговой системы заказчика известно, что данная система способна регистрировать до ХХХХХХ фактов оплат в час (ХХХХ в минуту).
Из п.~\ref{expGis} известно, что ГИС ЖКХ может обрабатывать до ХХХ фактов оплат в минуту.

\subsubsection*{План эксперимента}

\begin{enumerate}
	\item Согласование с заказчиком времени и продолжительности проведения нагрузочного тестирования.
	\item Настройка РИАС ЖКХ для проведения нагрузочного тестирования веб-сервиса платежей.
	\item Разработка программного обеспечения для проверки времени ответа веб-сервиса платежей РИАС ЖКХ.
	\item Сбор статистики по времени ответа веб-сервиса платежей РИАС ЖКХ.
	\item Сбор статистики по времени обработки фактов оплат для их отправки в ГИС ЖКХ.
	\item Сопоставление полученных данных с исходными.
	\item Составление выводов.
\end{enumerate}

\subsubsection*{Выполнение эксперимента}

TODO

\subsubsection*{Ожидаемые и полученные результаты}

TODO

\subsubsection*{Выводы}

TODO

\subsection{Сравнение скорости передачи данных различными методами обмена}

\subsubsection*{Цели и задачи эксперимента}

TODO

\subsubsection*{Исходные данные}

TODO

\subsubsection*{План эксперимента}

TODO

\subsubsection*{Выполнение эксперимента}

TODO

\subsubsection*{Ожидаемые и полученные результаты}

TODO

\subsubsection*{Выводы}

TODO

\clearpage
\newpage