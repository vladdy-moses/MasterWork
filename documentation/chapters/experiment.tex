\section{Описание экспериментов}

\subsection{Нагрузочное тестирование веб-сервиса платежей ГИС ЖКХ}

\subsubsection*{Цели и задачи эксперимента}

Целью данного вычислительного эксперимента является проверка корректности работы веб-сервиса регистрации платежей (фактов оплат) ГИС ЖКХ в момент пиковых нагрузок.

Данная цель достигается при помощи следующих задач:
\begin{enumerate}
	\item Внедрение механизма мониторинга времени обработки сообщений в ГИС ЖКХ для платёжного веб-сервиса;
	\item Получение от заказчиков контрольных значений по пиковым нагрузкам их биллинговых систем;
	\item Получение платежей (фактов оплат) в РИАС ЖКХ и их последующая отправка в ГИС ЖКХ в момент пиковой нагрузки;
	\item Сбор и обработка полученных временных интервалов обработки пакетов фактов оплат и аппроксимация результатов для большего числа платежей.
\end{enumerate}

\subsubsection*{Исходные данные}

Было получено разрешение от одного из заказчиков провести данный эксперимент на промышленных стендах РИАС ЖКХ и ГИС ЖКХ в рамках нагрузочных испытаний РИАС ЖКХ.

Время и продолжительность пика оплат за ЖКХ оказалось с ХХ:ХХ до ХХ:ХХ (TODO часов) и с ХХ:ХХ до ХХ:ХХ (TODO часов).
В это время каждую минуту у заказчика происходит около ХХХХ фактов оплат.

По требованию TODO приказа (TODO: ссылка) и технического задания на РИАС ЖКХ время отправки факта оплаты от его совершения до конца его регистрации в ГИС ЖКХ должно составлять 2 часа.

Необходимо узнать, будет ли превышен указанный интервал в конце пиковой нагрузки.

\subsubsection*{План эксперимента}

\begin{enumerate}
	\item Сбор данных о пиковых нагрузках. На этом этапе необходимо выяснить у заказчика (или разработчика его биллинговой системы), в какие промежутки времени происходит самый большой поток оплат за услуги ЖКХ.
	\item Разработка и внедрение механизма мониторинга времени обработки запроса в ГИС ЖКХ.
	\item Получение фактов оплат от заказчика при пиковой нагрузке.
	\item Регистрация фактов оплат в ГИС ЖКХ. На данном шаге следует отметить, что факты оплат в ГИС ЖКХ регистрируются пакетами от 1 до 1000 штук. Здесь следует отправить пакеты по 1000 платежей (пик), а также и более мелкие пакеты (полупик) для выявления зависимости числа времени обработки пакета от числа платежей в нём.
	\item Моделирование и аппроксимация полученных задержек регистрации платежей на весь период пика. На данном шаге вычисляется максимальная задержка платежа под конец пиковой нагрузки.
	\item Составление выводов о корректности работы веб-сервиса ГИС ЖКХ.
\end{enumerate}

\subsubsection*{Выполнение эксперимента}

TODO

\subsubsection*{Ожидаемые и полученные результаты}

TODO

\subsubsection*{Выводы}

TODO

\subsection{Нагрузочное тестирование веб-сервиса платежей РИАС ЖКХ}

\subsubsection*{Цели и задачи эксперимента}

TODO

\subsubsection*{Исходные данные}

TODO

\subsubsection*{План эксперимента}

TODO

\subsubsection*{Выполнение эксперимента}

TODO

\subsubsection*{Ожидаемые и полученные результаты}

TODO

\subsubsection*{Выводы}

TODO

\subsection{Сравнение скорости передачи данных различными методами обмена}

\subsubsection*{Цели и задачи эксперимента}

TODO

\subsubsection*{Исходные данные}

TODO

\subsubsection*{План эксперимента}

TODO

\subsubsection*{Выполнение эксперимента}

TODO

\subsubsection*{Ожидаемые и полученные результаты}

TODO

\subsubsection*{Выводы}

TODO

\clearpage
\newpage