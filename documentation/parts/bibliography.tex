\renewcommand{\refname}{Список использованных источников}

\begin{thebibliography}{99}
	\bibitem{gis_doc_tff} Альбом ТФФ v.11.2.0.8 / ГИС ЖКХ. – [Б. м. : б. и.], 2017. – Режим доступа: https://dom.gosuslugi.ru/\#!/regulation (дата обращения: 06.06.2017)
	\bibitem{gis_doc_tff1} Альбом ТФФ v.11.2.0.8 Приложение 1. Форматы электронных сообщений / ГИС ЖКХ. – [Б. м. : б. и.], 2017. – Режим доступа: https://dom.gosuslugi.ru/\#!/regulation (дата обращения: 06.06.2017)
	\bibitem{moiseev_org} Воронина В.В., Моисеев В.В., Организация взаимодействия модулей веб-ориентированного программного комплекса // Вузовская наука в современных условиях : сборник материалов 50-й научно-технической конференции. — Ульяновск: УлГТУ, 2016. — 271 с. — С. 244-247
	\bibitem{gost19701} ГОСТ 19.701-90. Единая система программной документации. Схемы алгоритмов, программ, данных и систем. Условные обозначения и правила выполнения. – М. : Стандартинформ, 2010.
	\bibitem{gost2105} ГОСТ 2.105-95. Единая система конструкторской документации. Общие требования к текстовым документам. – М. : Стандартинформ, 1996.
	\bibitem{gost3410} ГОСТ Р 34.10-2012. Информационная технология. Криптографическая защита информации. Процессы формирования и проверки электронной цифровой подписи. – М. : Стандартинформ, 2012. - Режим доступа: http://www.altell.ru/legislation/standards/gost-34.10-2012.pdf (дата обращения: 06.06.2017)
	\bibitem{gost34601} ГОСТ 34.601-90. Автоматизированные системы. Стадии создания. – М. : Изд-во стандартов, 1997.
	\bibitem{gost71} ГОСТ 7.1-2003. Система стандартов по информации, библиотечному и издательскому делу. Библиографическая запись. Библиографическое описание. Общие требования и правила составления.  М. : Стандартинформ, 2010.
	\bibitem{soap_dut} Дутиков, Д. Н. Применение SOAP-сервисов для обеспечения взаимодействия внутри распределённой информационной системы // Вестник Южно-Уральского государственного университета. - Челябинск, 2010. - №22 (198). - С. 9-14
	\bibitem{soap_zelen} Заленский, Д. А. Разработка универсальной модульной системы удалённой обработки геоданных, основанной на технологии SOAP (Simple Object Access Protocol) // Вестник Южно-Уральского государственного университета. - Ханты-Мансийск, 2006. - №4. - С. 31-35
	\bibitem{webProts} Кришнамурти Б., Рэксфорд Дж., Web-протоколы. Теория и практика. / Б. Кришнамурти, Дж. Рэксфорд – М.: ЗАО «Издательство БИНОМ», 2002. – 592 с.
	\bibitem{moiseev_saas} Куркина С.В., Моисеев В.В., Прохоров Е.Э и др., Анализ применимости SAAS-модели распространения программного обеспечения для автоматизации процессов ЖКХ // Прикладные информационные системы. — Ульяновск, 2016. — С. 240-245
	\bibitem{moiseev_kendo} Куркина С.В., Моисеев В.В., Прохоров Е.Э., Изучение возможностей платформы Telerik Kendo для отображения данных в веб-ориентированных информационных системах \\ // Прикладные информационные системы. — Ульяновск, 2016. — С. 232-239
	\bibitem{moiseev_meth} Моисеев В.В., Методы информационного взаимодействия информационных систем жилищно-коммунального хозяйства // Прикладные информационные системы. — Ульяновск, 2017
	\bibitem{gis_doc_test} Порядок проведения тестирования v.11.1.0.6 / ГИС ЖКХ. – [Б. м. : б. и.], 2017. – Режим доступа: https://dom.gosuslugi.ru/\#!/regulation (дата обращения: 06.06.2017)
	\bibitem{rest_plot} Плотников А. В. Реализация транзакций в RESTful веб-службах / А. В. Плотников, С. В. Золотарев // Актуальные проблемы прикладной математики, информатики и механики : сборник трудов междунар. науч.-тех. конференции, Воронеж, 12-15 сент. 2016 г. – Воронеж. : Изд-во ВГУ, 2016. – С. 165–167
	\bibitem {api_soap_rest} Поскребышев Р., Тарасов В., API на основе SOAP и REST / Р. Поскребышев, В. Тарасов // Молодые учёные - ускорению научно-технического прогресса в XXI веке : сборник материалов IV Всероссийской научно-технической конференции аспирантов, магистрантов и молодых учёных с международным участием. - Ижевск. : ИННОВА, 2016. - С. 404-410
	\bibitem{pr74114} Приказ Министерства связи и массовых коммуникаций РФ и Министерства строительства и жилищно-коммунального хозяйства РФ от 29 февраля 2016 г. № 74/114/пр <<Об утверждении состава, сроков и периодичности размещения информации поставщиками информации в государственной информационной системе жилищно-коммунального хозяйства>> / ГАРАНТ.РУ. – [Б. м. : б. и.], 2016. – Режим доступа: http://www.garant.ru/products/ipo/prime/doc/71311946/ (дата обращения: 06.06.2017)
	\bibitem{gis_doc_regul} Регламент взаимодействия внещних систем с ГИС ЖКХ v.11.1.0.7 / ГИС ЖКХ. – [Б. м. : б. и.], 2017. – Режим доступа: https://dom.gosuslugi.ru/\#!/regulation (дата обращения: 06.06.2017)
	\bibitem{rodionov} Родионов, В. В. Дипломное проектирование: учебно-методическое пособие для студентов специальности 23020165 <<Информационные системы и технологии>> / В. В. Родионов. – Ульяновск : УлГТУ, 2008. – 98 с.
	\bibitem{gis_doc_facets} Справочники ГИС ЖКХ v.11.2.0.8 / ГИС ЖКХ. – [Б. м. : б. и.], 2016. – Режим доступа: https://dom.gosuslugi.ru/\#!/regulation (дата обращения: 06.06.2017)
	\bibitem{troelsen} Троелсен, Э. Язык программирования C\# 2010 и платформа .NET 4 / Э. Троелсен. – 5-е изд. – М. : Вильямс, 2010. – 1392 с.
	\bibitem{moiseev_agile} Тронин В.Г., Моисеев В.В., Храмков Е.М., Использование принципов гибких методологий управления проектами при разработке региональной системы ЖКХ Ульяновской области // Вестник УлГТУ. — №2 (74). —Ульяновск, 2016. — С. 38-42
	\bibitem{fz209} Федеральный закон от 21 июля 2014 г. N 209-ФЗ <<О государственной информационной системе жилищно-коммунального хозяйства>> / Российская Газета. – [Б. м. : б. и.], 2014. – Режим доступа: https://rg.ru/2014/07/23/gkh-dok.html (дата обращения: 06.06.2017)
	\bibitem{fz188} Федеральный закон от 29.12.2004 N 188-ФЗ <<Жилищный кодекс Российской Федерации>> [Электронный ресурс] / КонсультантПлюс. – [Б. м. : б. и.], 2004. – Режим доступа: http://www.consultant.ru/document/cons\_doc\_LAW\_171389/ (дата обращения: 08.05.2015).
	\bibitem{flenov} Фленов, М. Е. Библия C\# / М. Е. Фленов. – 2-е изд. – СПб. : БХВ-Петербург, 2011. – 560 с.
	\bibitem{chertovkoy} Чертовской, В. Д. Базы и банки данных: Учебное пособие / В. Д. Чертовской. – СПб. : Изд-во МГУП, 2001. –  220 с.
	\bibitem{vs_mdn} Availability of Features in Visual Studio Versions // Microsoft Developer Network. – [Б. м. : б. и.], 2015. – Режим доступа: https://msdn.microsoft.com/en-us/library/ee519072.aspx (дата обращения: 24.05.2015)
	\bibitem{rest_rich} Richardson L., Amundsen M., Ruby S., RESTful Web APIs / Leonard Richardson, Mike Amundsen, Sam Ruby : O'Reilly Media, 2013. – 406 c.
	\bibitem{restBasics} Rogriguez, A. RESTful Web services: The basics / IBM developerWorks. - [Б. м. : б. и.], 2015. – Режим доступа: https://www.ibm.com/developerworks/library/ws-restful/ (дата обращения: 06.06.2017)
	\bibitem{soapSpec} SOAP Version 1.2 Part 1: Messaging Framework (Second Edition) / W3C Recommendation. - [Б. м. : б. и.], 2007. – Режим доступа: https://www.w3.org/TR/soap12-part1/ (дата обращения: 06.06.2017)
\end{thebibliography}

% \bibliographystyle{bibliography/ugost2003} % ГОСТ 7.1-2003
% \bibliography{bibliography/db} % db.bib
% Арестова, О. Н. Региональная специфика сообщества российских пользователей сети Интернет [Электронный ресурс] / О. Н. Арестова, Л. Н. Бабанин, А. Е. Войскунский. - Режим доступа: http://www.relarn.ru:8082/conf/conf97/10.html. - Загл. с экрана.

\clearpage
\newpage