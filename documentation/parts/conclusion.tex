\section*{Заключение}
\addcontentsline{toc}{section}{Заключение}

В рамках данной работы автор формально описал механизм информационного обмена, определил характеристики и ограничения типичных механизмов для сферы жилищно-коммунального хозяйства, доработал региональную информационного-аналитическую систему ЖКХ, провёл ряд вычислительных экспериментов в подтверждение выявленной модели и рекомендаций.

Работа является актуальной, так как в сфере ЖКХ до сих пор не существует единого формата информации, который необходимо заносить в единое место.
Также определённая предметная область является достаточно широкой, чтобы в ней могли сочетаться информационные системы различного назначения.

Работа над изучением информационного обмена в сфере ЖКХ заняла у автора более полутора лет.
Автор участвовал в технических сессиях по информационному обмену в сфере ЖКХ, проводимых Министерством Строительства Российской Федерации.

Работа имеет научные и инженерные перспективы.
Дальнейшая формализация принципов информационного обмена может быть полезна для приложения к методам обмена теории управления или анализа данных и систем.

Программные решения (два модуля региональной системы ЖКХ), созданные в рамках написания работы, были зарегистрированы как программы для ЭВМ.

Были успешно проведены вычислительные эксперименты, подтверждающие выделенные рекомендации по организации информационного обмена.
В частности, оценивалась скорость и объём передачи данных при помощи различных методов обмена на типовых наборах информации, а также применимость некоторых методов обмена при организации отправки платежей в ГИС ЖКХ.
Все вычислительные эксперименты успешно доказали рекомендации и применимость методов информационного обмена при промышленной эксплуатации.

\clearpage
\newpage