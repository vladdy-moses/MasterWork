\section*{Заключение}
\addcontentsline{toc}{section}{Заключение}

В рамках данной работы автор формально описал механизм информационного обмена, определил характеристики и ограничения типичных механизмов для сферы жилищно-коммунального хозяйства, доработал региональную информационного-аналитическую систему ЖКХ, провёл ряд вычислительных экспериментов в подтверждение выявленной модели и рекомендаций.

Работа является актуальной, так как в сфере ЖКХ до сих пор не существует единого формата информации, который необходимо заносить в единое место.
Также определённая предметная область является достаточно широкой, чтобы в ней могли сочетаться информационные системы разного назначения: справочные, расчётные, финансовые, аналитические и прочие.

Работа над изучением информационного обмена в сфере ЖКХ заняла у автора более полутора лет.
Более года автор работает над созданием модуля интеграции региональной системы ЖКХ с государственной информационной системой.
Автор участвовал в технических сессиях по информационному обмену в сфере ЖКХ, проводимых Министерством Строительства Российской Федерации и ЗАО <<ЛАНИТ>> (разработчик ГИС ЖКХ).

Работа имеет научные и инженерные перспективы.
Дальнейшая формализация принципов информационного обмена может быть полезна для приложения к методам обмена теории управления или анализа данных и систем.

Программные решения (два модуля региональной системы ЖКХ), созданные в рамках написания работы, были зарегистрированы как программы для ЭВМ.

\clearpage
\newpage