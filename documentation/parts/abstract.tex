\ESKDthisStyle{empty}

\section*{Аннотация}
%\addcontentsline{toc}{section}{Аннотация}

\textbf{Выпускная квалификационная работа} Моисеева Владислава Валерьевича по теме <<\WorkName>>.
Руководитель Воронина Валерия Вадимовна.
Защищена на кафедре <<Информационные системы>> УлГТУ в 2017 году.

\textbf{Пояснительная записка:} \ESKDtotal{page} с., 4 разд., \ESKDtotal{appendix} прил., \ESKDtotal{figure} рис., \ESKDtotal{table} TODO табл., \ESKDtotal{bibitem} ист.

\textbf{Ключевые слова:} информационный обмен,\linebreak жилищно-коммунальное хозяйство, РИАС ЖКХ, ГИС ЖКХ, SOAP, REST, SQL.

~

В данной выпускной квалификационной работе рассматривается вопрос организации информационного обмена между информационными системами и их модулями, разработанными в сфере жилищно-коммунального хозяйства.

В первой главе приводится описание рассматриваемых систем жилищно-коммунального хозяйства, определение и виды информационного обмена, а также указываются правовые основы информационного обмена в России.

Во второй главе приводится формализация модели информационного обмена, выделяются основные характеристики элементов модели, приводятся основные комбинации характеристик, применимых в сфере ЖКХ, а также рекомендации к их применению.

В третьей главе приводится описание дорабатываемой программной системы, где информационный процесс построен на основании выведенной модели.

В четвёртой главе описываются вычислительные эксперименты, которые позволяют сделать выводы о корректности применения различных механизмов информационного обмена к прикладным задачам.

\clearpage
\newpage