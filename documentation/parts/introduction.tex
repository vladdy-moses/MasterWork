\section*{Введение}
\addcontentsline{toc}{section}{Введение}

\subsection*{Краткое описание предметной области}
\addcontentsline{toc}{subsection}{Краткое описание предметной области}

Жилищно-коммунальное хозяйство Российской Федерации --- огромнейшая сфера экономики государства, которой присуща информационная неоднородность.
Множество компаний ведут свою деятельность в этой сфере, существует огромное множество нормативно-правовых актов, регулирующих деятельность таких организаций.

Повышение прозрачности в жилищно-коммунальном хозяйстве --- одна из важнейших целей государства.
Она достигается различными способами.
Одним из таких является обязанность некоторых типов организаций сферы ЖКХ раскрывать информацию о собственной деятельности в электронном виде.
Например, управляющие организации обязаны раскрывать информацию о финансово-хозяйственной деятельности, банки --- о поступивших платежах за ЖКХ, ресурсоснабжающие организации --- о договорах на поставку ресурсов.

Помимо поставщиков услуг и ресурсов, существуют организации и органы власти, регулирующие отношения и распределение финансовых потоков в сфере ЖКХ.
К таким можно отнести государственные жилищные инспекции (главрегионнадзоры или госжилинспекции) и фонды капитального ремонта.
Таким организациям требуется как изучать электронную отчётность поставщиков, так и самим отчитываться в электронной форме.

Для повышения прозрачности и раскрытия информации о деятельности организаций-участников ЖКХ было создано огромное множество информационных систем.
Такие информационные системы можно разделить на:
\begin{easylist}
& государственные;
& региональные;
& муниципальные;
& собственные.
\end{easylist}

Так как с каждым годом растёт нагрузка по раскрытию информации, были придуманы и реализованы механизмы информационного взаимодействия между информационными системами.
Они должны значительно упростить процесс раскрытия информации, так как до сих пор существует двойное (а иногда и тройное) дублирование информации в различные информационные системы.

\subsection*{Актуальность}
\addcontentsline{toc}{subsection}{Актуальность}

\textbf{То, что написано ниже, несомненно необходимо перечитать.}

Бурное развитие информационных технологий, несомненно, сказалось на увеличении комбинирования цифрового и материального пространств. 
Документооборот переходит в цифровую среду, как и подпись документов.
Многие книги учёта переводятся в электронную форму, так как с цифровой информацией легче работать.
Сфера жилищно-коммунального хозяйства не стала исключением.

На территории Ульяновской области в первом десятилетии XXI века был разработан ряд информационных систем для систематизации и упрощения работы в сфере жилищно-коммунального хозяйства.
В 2015-2016 годах эти информационные системы было решено объединить в РИАС ЖКХ субъекта Российской Федерации --- программный комплекс, охватывающий всю сферу ЖКХ региона внедрения.

Параллельно с этим процессом государство решило разработать государственную информационную систему жилищно-коммунального хозяйства (ГИС ЖКХ), описанную в федеральном законе 209-ФЗ от 21 июля 2014 года.
Данная информационная система предполагает хранение всей доступной информации о сфере ЖКХ [2]. Так как в РИАС ЖКХ вся необходимая информация уже есть, было решено настроить информационный обмен между системами.

Существуют также и другие информационные системы (государственные и муниципальные), с которыми необходимо настраивать взаимодействие в качестве региональной информационной системы жилищно-коммунального хозяйства.

Однако существующая на данный момент несогласованность некоторых данных в РИАС ЖКХ (отдельные системы разрабатывались параллельно около десятка лет) не даёт настроить обмен с ГИС ЖКХ и другими внешними системами корректно.

Данное исследование необходимо для понимания всей картины организации информационного обмена как между системами РИАС ЖКХ Ульяновской области, так и с внешними системами, с последующим описанием и реализацией выбранных методик обмена.

\subsection*{Научная новизна}
\addcontentsline{toc}{subsection}{Научная новизна}

В данной работе рассматриваются основные механизмы обмена данными между информационными системами сферы жилищно-коммунального хозяйства Российской Федерации при помощи телекоммуникационной сети Интернет.

Дополнительно в работе даются рекомендации по настройке информационного взаимодействия, приводится пример реализации такого взаимодействия с государственной информационной системой жилищно-коммунального хозяйства.

\subsection*{Положения, выносимые на защиту}
\addcontentsline{toc}{subsection}{Положения, выносимые на защиту}

TBD

\clearpage
\newpage