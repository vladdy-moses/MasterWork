\ESKDappendix{}{SQL-команды генерации данных для вычислительных экспериментов}

В текущем приложении приводится код на языке T-SQL, при помощи которого были сгенерированы тестовые наборы данных для проведения вычислительного эксперимента из п.~\ref{ext-echange}.

\subsubsection*{Набор данных $D_1$}
\begin{lstlisting}
-- Создание таблицы
IF 1=1 BEGIN
	CREATE TABLE sandbox.MoiseevTestSml (
		[TestSmlId] INT PRIMARY KEY IDENTITY(1,1),
		[ItemId] INT NOT NULL INDEX IX_TestSml_ItemId(TestSmlId) ,
		[Date] DATETIME2 NOT NULL,
		[Value] DECIMAL(18,4) NOT NULL
	)
END

-- Заполнение данными
IF 1=1 BEGIN
	CREATE TABLE #dates (d DATETIME2)
	CREATE TABLE #ids (i INT)

	;WITH cte AS (
		SELECT CAST('2017-01-01 00:00' AS DATETIME2) AS dt
		UNION ALL
		SELECT DATEADD(MINUTE, 1, dt)
		FROM cte s
		WHERE DATEADD(MINUTE, 1, dt) <= '2017-06-01 00:00')
	INSERT INTO #dates (d)
	SELECT s.dt
		FROM cte s
	OPTION (MAXRECURSION 0)

	;WITH cte AS (
		SELECT 1 AS dt
		UNION ALL
		SELECT 1+dt
		FROM cte s
		WHERE 1+dt <= 10)
	INSERT INTO #ids (i)
	SELECT s.dt
		FROM cte s
	OPTION (MAXRECURSION 0)

	INSERT INTO sandbox.MoiseevTestSml (ItemId, Date, Value)
	SELECT i.i, d.d, CAST(CAST(NEWID() AS VARBINARY) AS INT)
	FROM #dates d
		CROSS JOIN #ids i

	DROP TABLE #dates
	DROP TABLE #ids
END

-- Проверка числа записей
SELECT COUNT(*) FROM sandbox.MoiseevTestSml WITH(NOLOCK)
SELECT TOP 100 * FROM sandbox.MoiseevTestSml

-- Удаление таблицы
IF 1=0 BEGIN
	TRUNCATE TABLE sandbox.MoiseevTestSml
	DROP TABLE sandbox.MoiseevTestSml
END
\end{lstlisting}

\subsubsection*{Набор данных $D_2$}
\begin{lstlisting}
-- Создание таблицы
IF 1=1 BEGIN
	CREATE TABLE sandbox.MoiseevTestLarge (
		[TestLargeId] INT PRIMARY KEY IDENTITY(1,1),
		[Date] DATETIME2 NOT NULL,
		[Value01] DECIMAL(18,4) NOT NULL,
		[Value02] DECIMAL(18,4) NOT NULL,
		[Value03] DECIMAL(18,4) NOT NULL,
		[Value04] DECIMAL(18,4) NOT NULL,
		[Value05] NVARCHAR(MAX) NOT NULL,
		[Value06] NVARCHAR(MAX) NOT NULL,
		[Value07] NVARCHAR(MAX) NOT NULL,
		[Value08] NVARCHAR(MAX) NOT NULL,
		[Value09] UNIQUEIDENTIFIER NOT NULL,
		[Value10] UNIQUEIDENTIFIER NOT NULL,
		[Value11] INT NOT NULL,
		[Value12] INT NOT NULL,
		[Value13] INT NOT NULL,
		[Value14] INT NOT NULL
	)
END

-- Заполнение данными
IF 1=1 BEGIN
	CREATE TABLE #dates (d DATETIME2)
	CREATE TABLE #ids (i INT)

	;WITH cte AS (
		SELECT CAST('2017-01-01 00:00' AS DATETIME2) AS dt
		UNION ALL
		SELECT DATEADD(MINUTE, 5, dt)
		FROM cte s
		WHERE DATEADD(MINUTE, 5, dt) <= '2017-06-01 00:00')
	INSERT INTO #dates (d)
	SELECT s.dt
		FROM cte s
	OPTION (MAXRECURSION 0)

	;WITH cte AS (
		SELECT 1 AS dt
		UNION ALL
		SELECT 1+dt
		FROM cte s
		WHERE 1+dt <= 10)
	INSERT INTO #ids (i)
	SELECT s.dt
		FROM cte s
	OPTION (MAXRECURSION 0)

	INSERT INTO sandbox.MoiseevTestLarge
	        ( Date ,
	          Value01 ,
	          Value02 ,
	          Value03 ,
	          Value04 ,
	          Value05 ,
	          Value06 ,
	          Value07 ,
	          Value08 ,
	          Value09 ,
	          Value10 ,
	          Value11 ,
	          Value12 ,
	          Value13 ,
	          Value14
	        )
	SELECT d.d
		, CAST(CAST(NEWID() AS VARBINARY) AS INT), CAST(CAST(NEWID() AS VARBINARY) AS INT), CAST(CAST(NEWID() AS VARBINARY) AS INT), CAST(CAST(NEWID() AS VARBINARY) AS INT)
		, CAST(NEWID() AS NVARCHAR(MAX)), CAST(NEWID() AS NVARCHAR(MAX)), CAST(NEWID() AS NVARCHAR(MAX)), CAST(NEWID() AS NVARCHAR(MAX))
		, NEWID(), NEWID()
		, CAST(CAST(NEWID() AS VARBINARY) AS INT), CAST(CAST(NEWID() AS VARBINARY) AS INT), CAST(CAST(NEWID() AS VARBINARY) AS INT), CAST(CAST(NEWID() AS VARBINARY) AS INT)
	FROM #dates d
		CROSS JOIN #ids i

	DROP TABLE #dates
	DROP TABLE #ids
END

-- Проверка числа записей
SELECT COUNT(*) FROM sandbox.MoiseevTestLarge WITH(NOLOCK)
SELECT TOP 100 * FROM sandbox.MoiseevTestLarge

-- Удаление таблицы
IF 1=0 BEGIN
	TRUNCATE TABLE sandbox.MoiseevTestLarge
	DROP TABLE sandbox.MoiseevTestLarge
END
\end{lstlisting}

\subsubsection*{Набор данных $D_3$}
\begin{lstlisting}
-- Создание таблиц
IF 1=1 BEGIN
	CREATE TABLE sandbox.MoiseevTestMany1 (
		[Id] INT PRIMARY KEY IDENTITY(1,1),
		[Date] DATETIME2 NOT NULL,
		[Value01] DECIMAL(18,4) NOT NULL,
		[Value02] DECIMAL(18,4) NOT NULL,
		[Value03] DECIMAL(18,4) NOT NULL,
		[Value04] DECIMAL(18,4) NOT NULL,
		[Value05] NVARCHAR(MAX) NOT NULL,
		[Value06] NVARCHAR(MAX) NOT NULL,
		[Value07] NVARCHAR(MAX) NOT NULL,
		[Value08] NVARCHAR(MAX) NOT NULL
	)

	CREATE TABLE sandbox.MoiseevTestMany2 (
		[Id] INT PRIMARY KEY IDENTITY(1,1),
		[Many1Id] INT NOT NULL,
		[Value01] UNIQUEIDENTIFIER NOT NULL,
		[Value02] UNIQUEIDENTIFIER NOT NULL
	)

	CREATE TABLE sandbox.MoiseevTestMany3 (
		[Id] INT PRIMARY KEY IDENTITY(1,1),
		[Many1Id] INT NOT NULL,
		[Value01] INT NOT NULL,
		[Value02] INT NOT NULL,
		[Value03] INT NOT NULL,
		[Value04] INT NOT NULL
	)

	ALTER TABLE sandbox.MoiseevTestMany2 ADD CONSTRAINT MoiseevMany2_FK FOREIGN KEY (Many1Id) REFERENCES sandbox.MoiseevTestMany1(Id)
	ALTER TABLE sandbox.MoiseevTestMany3 ADD CONSTRAINT MoiseevMany3_FK FOREIGN KEY (Many1Id) REFERENCES sandbox.MoiseevTestMany1(Id)
	CREATE NONCLUSTERED INDEX IX_MoiseevMany2 ON sandbox.MoiseevTestMany2(Many1Id)
	CREATE NONCLUSTERED INDEX IX_MoiseevMany3 ON sandbox.MoiseevTestMany3(Many1Id)
END

-- Заполнение данными
IF 1=1 BEGIN
	CREATE TABLE #dates (d DATETIME2)
	CREATE TABLE #ids (i INT)

	;WITH cte AS (
		SELECT CAST('2017-01-01 00:00' AS DATETIME2) AS dt
		UNION ALL
		SELECT DATEADD(MINUTE, 5, dt)
		FROM cte s
		WHERE DATEADD(MINUTE, 5, dt) <= '2017-06-01 00:00')
	INSERT INTO #dates (d)
	SELECT s.dt
		FROM cte s
	OPTION (MAXRECURSION 0)

	;WITH cte AS (
		SELECT 1 AS dt
		UNION ALL
		SELECT 1+dt
		FROM cte s
		WHERE 1+dt <= 10)
	INSERT INTO #ids (i)
	SELECT s.dt
		FROM cte s
	OPTION (MAXRECURSION 0)

	INSERT INTO sandbox.MoiseevTestMany1
	        ( Date ,
	          Value01 ,
	          Value02 ,
	          Value03 ,
	          Value04 ,
	          Value05 ,
	          Value06 ,
	          Value07 ,
	          Value08
	        )
	SELECT d.d
		, CAST(CAST(NEWID() AS VARBINARY) AS INT), CAST(CAST(NEWID() AS VARBINARY) AS INT), CAST(CAST(NEWID() AS VARBINARY) AS INT), CAST(CAST(NEWID() AS VARBINARY) AS INT)
		, CAST(NEWID() AS NVARCHAR(MAX)), CAST(NEWID() AS NVARCHAR(MAX)), CAST(NEWID() AS NVARCHAR(MAX)), CAST(NEWID() AS NVARCHAR(MAX))
	FROM #dates d
		CROSS JOIN #ids i

	INSERT INTO sandbox.MoiseevTestMany2
	        ( Many1Id, Value01, Value02 )
	SELECT m1.Id, NEWID(), NEWID()
	FROM sandbox.MoiseevTestMany1 m1
		CROSS JOIN #ids i

	INSERT INTO sandbox.MoiseevTestMany3
	        ( Many1Id ,
	          Value01 ,
	          Value02 ,
	          Value03 ,
	          Value04
	        )
	SELECT m1.Id, i.i+1, i.i+2, i.i+3, i.i+4
	FROM sandbox.MoiseevTestMany1 m1
		CROSS JOIN #ids i

	DROP TABLE #dates
	DROP TABLE #ids
END

-- Проверка числа записей
SELECT COUNT(*) FROM sandbox.MoiseevTestMany1 WITH(NOLOCK)
SELECT COUNT(*) FROM sandbox.MoiseevTestMany2 WITH(NOLOCK)
SELECT COUNT(*) FROM sandbox.MoiseevTestMany3 WITH(NOLOCK)

-- Удаление таблиц
IF 1=0 BEGIN
	TRUNCATE TABLE sandbox.MoiseevTestMany1
	TRUNCATE TABLE sandbox.MoiseevTestMany2
	TRUNCATE TABLE sandbox.MoiseevTestMany3
	DROP TABLE sandbox.MoiseevTestMany1
	DROP TABLE sandbox.MoiseevTestMany2
	DROP TABLE sandbox.MoiseevTestMany3
END
\end{lstlisting}